%%%%%%%%%%%%%%%%%%%%%%%%%%%%%%%%%%%%%%%%%%%%%%%%%%%%%%%%%%%%%%%%%%%%%%%%
% Requires the installation of Microsoft fonts (Arial).
% Install on Linux with "sudo apt-get install ttf-mscorefonts-installer"
%%%%%%%%%%%%%%%%%%%%%%%%%%%%%%%%%%%%%%%%%%%%%%%%%%%%%%%%%%%%%%%%%%%%%%%%

\documentclass[11pt,a4paper]{article}

%%%%%%%%%%%%%%%%%%%%  packages and definitions %%%%%%%%%%%%%%%%%%%%%%%%%

\usepackage[margin=2cm]{geometry}

\usepackage{url}
\usepackage{cite}
\usepackage{xcolor}
\usepackage{tabularx}
\usepackage{siunitx}
\usepackage{setspace}
\setstretch{1.0} 
\usepackage{titlesec}
\usepackage{enumitem}
\setlist[itemize]{itemsep=4pt,parsep=2pt,topsep=2pt,leftmargin=0.5cm,labelsep=4pt}

% setting Arial font
\usepackage{fontspec}
\usepackage{fontspec-luatex}
\setmainfont{Arial} 
% for Gantt charts
\usepackage{pgfgantt}

\usepackage{cleveref}

% colors
\definecolor{barblue}{RGB}{153,204,254}
\definecolor{groupblue}{RGB}{51,102,254}
\definecolor{linkred}{RGB}{165,0,33}
\definecolor{linkgreen}{RGB}{20,182,117}
\definecolor{linkorange}{RGB}{205,134,41}
\definecolor{green}{RGB}{0, 128, 0}
\definecolor{red}{RGB}{255, 0, 0}
\definecolor{lightblue}{RGB}{173, 216, 230}
\definecolor{mydarkblue}{RGB}{0, 0, 139}
\definecolor{mydarkgreen}{RGB}{0, 100, 0}
\definecolor{mylightgreen}{RGB}{56, 93, 75}
\setganttlinklabel{s-s}{START-TO-START}
\setganttlinklabel{f-s}{FINISH-TO-START}
\setganttlinklabel{f-f}{FINISH-TO-FINISH}
\sffamily

\usepackage{array}
\newcolumntype{P}[1]{>{\raggedright\arraybackslash}p{#1}}
\usepackage{multirow}
\usepackage{pifont}
\usepackage{wasysym}
\usepackage{pstricks}
\usepackage{wasysym}

\author{Your Name}
\title{Bioengineering and Robotics PhD annual report template}

\titleformat{\section}
{\bfseries\normalsize} % Formatting for the title (bold and large)
{\thesection.}     % Section number
{0.5em}        
     % Space between section number and title
{}                % Code before the title

\titleformat{\subsection}
{\normalsize} % Formatting for the title (bold and large)
{\thesubsection}     % Section number
{1em}        
% Space between section number and title
{}         

\begin{document}
\begin{figure}[t]
	\centering
	\includegraphics[width=0.99\textwidth]{logos/top_logos.pdf}
\end{figure}
\hphantom{h}\vspace{1.3cm}


\begin{center}
	
\LARGE{\textbf{Doctorate in Bioengineering and Robotics }\vspace{0.4cm}\\
Curriculum: [curriculum name]}

\end{center}

\begin{flushleft}
\vspace{1.5cm}
\large{\textbf{Student: }}\large{{[surname], [name]}}\vspace{0.2cm}\\
\large{\textbf{Cycle: }}\large{[cycle number]}\vspace{0.2cm}\\
\large{\textbf{Tutor(s): }}\large{{[surname], [name]}}\vspace{0.2cm}\\
\large{\textbf{Year: }}\large{[academic year]}\vspace{0.2cm}\\
\large{\textbf{Contacts: }}\large{{my.email@me.it}}
\end{flushleft}
\clearpage
\begin{center}
	\textbf{\normalsize{Title of your research project\\ (tentative)}}
\end{center}
\section{Objectives [max 1 page]}
Motivate your research and define the general goal of the research project (i.e., which is the addressed S/T question? \\
Detail the specific objectives of the current year, and summarize the planned activities of the other two/one years.  

\section{State of art and proposed innovation [max 1 page, max 20 refs]} 
Properly frame the research in the literature. Which is the gap you plan to bridge? Which are the open issues you aim to address? 
(If the objectives of the project do not change during the 3-years, this section could be filled only the first year and modified only by adding new relevant works published). 

\section{Methodology and workplan [max 1.5 page]}
Provide: 
\begin{itemize}
\item[-] Overview of the proposed approach you plan to follow (including methods and techniques)
\item[-] List of tasks organized in tables (see below). (For each task, brief summary of the progress of the research accomplishments (the details about the results achieved will be provided in Sect. 4). 

\begin{table}[h]
	\begin{center}
	\renewcommand{\arraystretch}{1.3} % Adjust the vertical spacing
	\setlength{\tabcolsep}{8pt} % Adjust the horizontal spacing
	\begin{tabular}{|P{15cm}|}
		\hline
		\textbf{Task name: ......................} \\ \hline
		\textbf{Scheduling: }(e.g. month 1-12) \\ \hline
		\textbf{Performed actions: }(3-4 lines)\\
		\hline
		\textbf{Status: }(e.g.: “successfully completed by month 12” or “the work has \\ started and proceeds as planned“ or “delay due to….”)\\
		\hline
		\textbf{Publications relative to the task: } (referring to the list in Sect. 6)\\
		\hline
		\textbf{Revised planning: }\\
		\hline
	\end{tabular}
\end{center}
\end{table}
Additionally, a Gantt chart is required (see below for an example).
\begin{figure}[h]
	\begin{center}
		\begin{ganttchart}[
			canvas/.append style={fill=none, draw=black!5, line width=.75pt},
			hgrid style/.style={draw=black!40, line width=1.0pt},
			vgrid={*1{draw=black!40, line width=1.0pt}},
			today=11,
			today rule/.style={
				draw=black,
				dash pattern=on 3.5pt off 4.5pt,
				line width=1.8pt
			},
			y unit chart=1cm,
			today label font=\footnotesize\bfseries,
			title/.style={text=black, draw=black, fill=gray!30, font=\Large\bfseries, align=center},
			title label font=\bfseries\footnotesize,
			title label node/.append style={below=-7pt},
			title left shift=0.0,
			title top shift=0.2,
			include title in canvas=false,
			bar label font=\mdseries\small\color{black!90},
			bar/.append style={draw=none, fill=blue!40},
			bar incomplete/.append style={fill=lightblue!80},
			bar height=0.2,
			bar label node/.append style={left=0.2cm, fill=black!10},
			bar progress label font=\bfseries\small,
			bar progress label font=\mdseries\footnotesize\color{black!80},
			%	progress label text={\si{\percent}},
			group incomplete/.append style={fill=lightblue},
			group/.append style={
				fill=mydarkgreen!60, % Change this color for completed group bars
			},
			group left shift=0,
			group right shift=0,
			group label font=\mdseries\bfseries\small\color{black!90},
			group height=.3,
			group peaks tip position=0,
			group label node/.append style={left=0.4cm, fill=black!20},
			group progress label font=\mdseries\footnotesize\color{black!80},
			]{1}{12}
			\gantttitle[
			title label node/.append style={below left=-7pt and -7pt}
			]{Months:\quad1}{1}
			\gantttitlelist{2,...,12}{1} \\
			[grid]
			\ganttgroup[progress=61]{Macro-task 1}{1}{10} \\
			[grid]
			\ganttbar[
			progress=75,
			name=subtask1
			]{\textbf{Subtask 1.1}}{1}{8} \\
			[grid]
			\ganttbar[
			progress=47,
			name=subtask1
			]{\textbf{Subtask 1.2}}{4}{10}
		\end{ganttchart}
	\end{center}
	\caption{Example Gantt chart for your tasks.}
\end{figure}

\end{itemize}
\section{Results in the reporting period [max 3 pages including figures]}
(interim highlights)  \\
A short paragraph (e.g. 200-300 characters) for presenting the interim results in the global picture. Details of the interim results (e.g. highlights about specific methodology or specific results) 
\section{Training}
List the courses, schools, training activities followed during this academic year.
Example:
\begin{table}[h]
	\begin{center}
		\renewcommand{\arraystretch}{1.3} % Adjust the vertical spacing
		\setlength{\tabcolsep}{8pt} % Adjust the horizontal spacing
		\begin{tabular}{|P{6cm}|P{3cm}|c|P{3cm}|P{0.8cm}|}
			\hline
			\textbf{Activity name} & \textbf{Type} & \textbf{Status} & \textbf{Location} & \textbf{CF} \\ \hline
			Ethics and Bioethics in Bioengineering and Robotics (Prof. Linda Battistuzzi) & PhD course (mandatory) & \textcolor{mydarkgreen}{\ding{51}} & UniGe - DIBRIS & 5 \\ \hline 
			Paper Writing (Prof. Mario Marchese) & PhD course (mandatory)& \textcolor{mydarkgreen}{\ding{51}} & UniGe - DIBRIS & 5 \\ \hline 
			Open Science and Research Data Management (OS\&RDM, Prof. Anna Maria Pastorini) & PhD course (mandatory)& \textcolor{mydarkgreen}{\ding{51}} & UniGe - DIBRIS & 4 \\ \hline 
			Grant Writing (Dr. Cinzia Leone)& PhD course (mandatory)& \textcolor{mydarkblue}{\VHF} & UniGe - DIBRIS & 5 \\ \hline 
			Theatrical techniques for public speaking (Prof. Antonio Sgorbissa) & PhD course (mandatory)& \textcolor{red}{\ding{55}} & UniGe - DIBRIS & 5 \\ \hline 
			\multicolumn{4}{|c|}{Partial mandatory credits} & 14/24\\ \hline
			\multicolumn{4}{|c|}{Still to be acquired} & 10\\ \hline
			Modern C++ (Marco Accame) & PhD course (optional)& \textcolor{mydarkgreen}{\ding{51}} & UniGe - DIBRIS & 9 \\ \hline 
			Robot Behaviour Modelling (Michele Colledanchise) & PhD course (optional)& \textcolor{mydarkgreen}{\ding{51}} & UniGe - DIBRIS & 4 \\ \hline 
			Beautiful Summer School.  & Summer school & \textcolor{mydarkgreen}{\ding{51}} & Hawaii & 3 \\ \hline 
			\multicolumn{4}{|c|}{Partial optional credits} & 16/16\\ \hline
			\multicolumn{4}{|c|}{Still to be acquired} & 0\\ \hline
		\end{tabular}
	\end{center}
	\caption{Example report of training activities. Symbol \textcolor{mydarkgreen}{\ding{51}} sta ds for completed, \textcolor{red}{\ding{55}} not started, \textcolor{mydarkblue}{\VHF} in progress.}
\end{table}
\section{Publication record}
\subsection{Peer reviewed journal papers}
\begin{itemize}
	\item[-] \cite{dummy_journal1}
	\item[-] \cite{dummy_journal2}
\end{itemize}
\subsection{Peer-reviewed conference proceedings}
\begin{itemize}
	\item[-] \cite{dummy_conference1}
\end{itemize}
\subsection{Book-chapters}
\begin{itemize}
	\item[-] .....
\end{itemize}
\section{Other activities}
For example:
\begin{itemize}
	\item[a.] Presentations
	\item[b.] Attended conferences, workshops, etc 
	\item[c.] Teaching
	\item[d.] .......
\end{itemize}
  
\cleardoublepage

\bibliographystyle{unsrt}
\bibliography{refs/refs.bib}

\end{document}
